% -*- coding: utf-8 -*-
% !TEX program = xelatex
\documentclass{jnuthesis}

\begin{document}

\renewcommand{\title}{An Overview of Morse Theory: Slicing Technique in Manifolds} % 英文标题
\renewcommand{\biaoti}{\songti \zihao{-3}Morse理论综述:流形中的切片技术}  % 中文标题
\renewcommand{\xueyuan}{\songti \zihao{-3} 暨南大学伯明翰大学联合学院}
\renewcommand{\xuexi}{\;}
\renewcommand{\zhuanye}{\songti \zihao{-3}数学与应用数学}
\renewcommand{\xingming}{\songti \zihao{-3}陈渝硕}
\renewcommand{\xuehao}{\songti \zihao{-3}2020101918}
\renewcommand{\daoshi}{\songti \zihao{-3}赫海龙}

\renewcommand{\name}{Chen Yushuo}
\renewcommand{\stuno}{2020101918}
\renewcommand{\major}{Mathematics And Apply Mathematics}
\renewcommand{\supervisor}{Her-Hailong}

\begin{titlepage}
  
  \thispagestyle{empty}
  % \centerline{\includegraphics[width=4cm,height=4cm]{jnulogo}}
  \vspace*{\fill}
  \begin{center}
  \zihao{1}\heiti
  暨\ 南\ 大\ 学 \\ 本\ 科\ 生\ 毕\ 业\ 论\ 文
  \end{center}
  \vfill
  \noindent {\songti \zihao{-3}论文题目}:\fillout{\biaoti}
  \vfill
  \begin{center}
    \begin{tabular}{l}
  \textbf{\songti \zihao{-3}学\qquad 院}:\underbox{18em}{\xueyuan}\\[0.6em]
  \textbf{\songti \zihao{-3}学\qquad 系}:\underbox{18em}{\xuexi}\\[0.6em]
  \textbf{\songti \zihao{-3}专\qquad 业}:\underbox{18em}{\zhuanye}\\[0.6em]
  \textbf{\songti \zihao{-3}姓\qquad 名}:\underbox{18em}{\xingming}\\[0.6em]
  \textbf{\songti \zihao{-3}学\qquad 号}:\underbox{18em}{\xuehao}\\[0.6em]
  \textbf{\songti \zihao{-3}指导教师}:\underbox{18em}{\daoshi}
    \end{tabular}
  \end{center}
  \vspace{2em}
  \centerline{\today}
  \newpage
\end{titlepage}


% \titlepage

\statement

\begin{enabstract}
\abstract
A concise abstract should briefly state the purpose of the research and the main results. 
An abstract is often presented separate from the article, so it must be able to stand alone. 
The abstract should be about 150 words with 3-8 key words.
\keywords
One, Two, Three, Four
\end{enabstract}

\tableofcontents

\chapter{Notation and Conventions}
符号索引
术语
\begin{itemize}
  \item abc
\end{itemize}


\chapter{Introduction}


\chapter{part 1}
\section{1}

$\cdots\cdots$


\chapter{Conclusion}

\appendix

% 2025.3.6  禁用  论文要求结论不作为单独一章
% \chapter*{Conclusion}\addcontentsline{toc}{chapter}{\hspace{-1em}Conclusion}
% /*结论作为单独一章排列,但不加章号。
% 结论是对整个论文主要成果的归纳,要突出设计(论文)的创新点,
% 以简练的文字对论文的主要工作进行评价,一般为400~1 000字。*/

\chapter*{\centerline{\large Acknowledgements}}\addcontentsline{toc}{chapter}{\hspace{-1em}Acknowledgements}
感谢我的导师XXX老师,谢谢他对我的悉心指导。
他无私的关爱和严谨的治学态度,将激励我不断的进取,走好以后的道路。
其次,还要感谢在这四年的学习中教过我的所有老师们,谢谢他们传授给了我知识。
我的同学XXX,在写作的过程中给我提供了一些宝贵的资料和建议,在此一并感谢!

\chapter*{\centerline{\large Appendix}}\addcontentsline{toc}{chapter}{\hspace{-1em}Appendix}
/*是正文主体的补充项目,并不是必需的。下列内容可以作为附录:
(1)为了整篇材料的完整,插入正文又有损于编排条理性和逻辑性的材料;
(2)由于篇幅过大,或取材于复制件不便编入正文的材料;
(3)对一般读者并非必须阅读,但对本专业人员有参考价值的资料;
(如外文文献复印件及中文译文、公式的推导、程序流程图、图纸、数据表格等)
附录按“附录A,附录B,附录A1“等编号。
请单击样式“附录1”为第1级的附录编号,样式“附录2”为第二级的附录编号,样式“附录3”控制第三级别的样式。*/

\chapter*{\centerline{\large References}}\addcontentsline{toc}{chapter}{\hspace{-1em}References}
引用文献标示应置于所引内容最末句的右上角。所引文献编号用阿拉伯数字置于方括号“[ ]”中,如“二次铣削[1]”。
如同一处引用了多个文献,文献编号间用逗号分隔,如“二次铣削[1,3] ”。
当提及的参考文献为文中直接说明时,其序号应该与正文排齐,如“由文献[8,10~14]可知”。
经济、管理类论文引用文献,若引用的是原话,要加引号,一般写在段中;若引的不是原文只是原意,
文前只需用冒号或逗号,而不用引号。在参考文献之外,若有注释的话,建议采用夹注,即紧接文句,用圆括号标明。
或者以脚注的形式排在页面底端,按①,②,③编号。

\end{document}
