\documentclass{report}
\usepackage{caption}
\usepackage{setspace}
\usepackage{amsmath}
\usepackage{amssymb}
\usepackage{amsthm}
\usepackage{graphicx}
\usepackage{multirow}
\usepackage{geometry}
\usepackage{listings}
\usepackage{subfigure}
\usepackage{float}
% \geometry{a4paper,scale=0.8}

% \begin{spacing}{1.3541667}
% \end{spacing}

\begin{document}
\begin{spacing}{1.3541667}

\begin{center}
\section*{Question 1}
\end{center}

\par By the definition in the textbook, we know that the effective (absolute) interest rate 
of a given investment of duration $T =10$ years is the rate R such that

\begin{align*}
    \boxed{{(1+r\delta t)}^{\frac{T}{\delta t}}I_0=I_F=(1+R)I_0}
\end{align*}

Then we know that $I_0=11$ Gils, $I_F=15$ Gils and $T=10$ years, we can get (with 2 decimal digits)

\begin{align*}
    \boxed{R=0.36}
\end{align*}

So the absolute interest rate is 0.36.

\newpage

\begin{center}
    \section*{Question 2}
\end{center}

\par We know the value of bank account with the nominal interest rate r and compounding period $\delta t$ is 

\begin{align*}
    \boxed{C_t={(1+r\delta t)}^{\frac{t}{\delta t}}C_0}
\end{align*}

With $r=2$ \%.year$^{-1}$, $\delta t=6$ months $=0.5$ year, $C_0=10$ Gils and $t=1$ year. Then we get (with 0 decimal digit)

\begin{align*}
    \boxed{C_t =10\;Gils}
\end{align*}

\par So the value of my bank account when I close it which is 1 year later is 10 Gils (with 0 decimal digit). 
That means I almost cannot get any profit from the bank. 
And if it is necessary to count with the profit, it is too small respect to $C_0=10$ Gils. 

\newpage

\begin{center}
    \section*{Question 3}
\end{center}

\par This question require us to calculate the present value. 
According to the formula 

\begin{align*}
    \boxed{C_t={(1+r\delta t)}^{\frac{t}{\delta t}}C_0}
\end{align*}

This time we have $C_t=10$ Gils, $r=1\%$ per period and $\delta t =1$ months $=\frac{1}{12}$ year. 
Then the result is (with 2 decimal digits)

\begin{align*}
    \boxed{C_0=9.80\;Gils}
\end{align*}

\par That means $C_0=9.80$ Gils is the value $C_2=9.80$ Gils when 2 months after. 


\newpage

\begin{center}
    \section*{Question 4}
\end{center}

\par We can calculate the value of bank account after $t=1$ year buying this two product represently. 
So we need to use the formula showed previously which is 

\begin{align*}
    \boxed{C_t={(1+r\delta t)}^{\frac{t}{\delta t}}C_0}
\end{align*}

Since $t=t^A=t^B=1$ year, $C_0=10$ Gils. Next we calculate the value represently. 

\par 1. Bank A

We can get that (with 3 decimal digits) 

\begin{align*}
    C^{A}_t&={(1+1\%\cdot \frac{1}{6})}^{6} \cdot C_0\\
    &=1.01004 \times 10\\
    &=10.100\;Gils
\end{align*}

\par 2. Bank B

We can get that (with 3 decimal digits) 

\begin{align*}
    C^{B}_t&={(1+2\%\cdot \frac{1}{4})}^{4} \cdot C_0\\
    &=1.02015 \times 10\\
    &=10.202\;Gils
\end{align*}

So we can find $\boxed{C^A_t=10.100\;Gils}$, $\boxed{C^B_t=10.202\;Gils}$. 
Camparing this 2 value, we can easily find that $C^A_t<C^B_t$. 
So the product in bank B can get more value than in bank A. 
We can claim that the product in bank B is more advantageous. 

\newpage

\begin{center}
    \section*{Question 5}
\end{center}

\par According to the payoff formula about American call option which is 

\begin{align*}
    \boxed{\xi=g(X_T)={(K-X_T)}^+}
\end{align*}

If I exercise my option now with current price $X_0=100$ Gils and strike $K=90$ Gils 
I can get the payoff 

\begin{align*}
    \boxed{\xi_0 = 10\;Gils}
\end{align*}

\par Since the derivative is an American call option, and we can exercise is at any time, 
we can get the payoff $\xi_0 =10$ Gils if we exercise now at $t=0$. 









\end{spacing}

\end{document}